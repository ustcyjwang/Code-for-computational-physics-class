%!TEX program = xelatex
\documentclass{article}
\usepackage[UTF8]{ctex}
\usepackage{authblk}
\usepackage{listings}
\usepackage{subfigure}
\usepackage{amsmath, amsfonts, amsthm, amssymb, graphicx, xcolor, hyperref, mathrsfs}
\lstset{
    basicstyle          =   \sffamily,          % 基本代码风格
    keywordstyle        =   \bfseries,          % 关键字风格
    commentstyle        =   \rmfamily\itshape,  % 注释的风格,斜体
    stringstyle         =   \ttfamily,  % 字符串风格
    flexiblecolumns,                % 别问为什么,加上这个
    numbers             =   left,   % 行号的位置在左边
    showspaces          =   false,  % 是否显示空格,显示了有点乱,所以不现实了
    numberstyle         =   \zihao{-5}\ttfamily,    % 行号的样式,小五号,tt等宽字体
    showstringspaces    =   false,
    captionpos          =   t,      % 这段代码的名字所呈现的位置,t指的是top上面
    frame               =   lrtb,   % 显示边框
    breaklines%自动换行
}

\title{计算物理作业8} %文章标题
\author[a]{王一杰} %作者的名称
\affil[a]{中国科学技术大学}
\date{\today}%日期
\usepackage [a4paper, left=25mm,right=25mm,top=20mm,bottom=20mm]{geometry} %设置页面的环境,a4纸张大小,左右上下边距信息

\begin{document}
\maketitle
%添加这一句才能够显示标题等信息
\tableofcontents

\section{Problem 1}
\subsection{程序设定}
\textbf{问题描述:}

一个小球质量$m$,以初速度$V_0$,与水平方向夹角$\theta$,从坐标原点开始做抛物运动。它在空气中的阻力可以近似用$-b\overrightarrow{v}$来 表示,$\overrightarrow{v}$为速度矢量。使用mathematica求解运动轨迹问题。

\textbf{程序设定:}

\quad 小球质量$m=0.14kg$,

\quad 重力加速度$g=9.8m/s^2$,

\quad 初速率$V_0=45m/s$,

\quad 出射角$\theta=\frac{\pi}{3}$.

\subsection{程序实现}

程序实现如下:

\begin{lstlisting}[language=Mathematica]
Clear["Global`*"]
sol = DSolve[{x''[t] == -(b/m)*x'[t], y''[t] == -(b/m)*y'[t] - g, 
     x[0] == 0, y[0] == 0, x'[0] == v0*Cos[\[Theta]], 
     y'[0] == v0*Sin[\[Theta]]}, {x[t], y[t]}, t] // ExpandAll;
(*运动方程求解*)
x[t_] = (x[t] /. Flatten[sol][[1]]);
y[t_] = (y[t] /. Flatten[sol][[2]]);
t[t_] = InverseFunction[x][t] // Simplify;
(*反函数,利用x表示t*)
yx[x_] = y[t[x]]
(*t(x)带入y(t),获得表达式y(x)*)

yyx[x_] = 
 yx[x] //. {b -> 0.033, m -> 0.14, v0 -> 45, \[Theta] -> \[Pi]/3, 
   g -> 9.8}
(*赋值*)
yy0[x_] = 
  Tan[\[Theta]]*x - (1/2)*(g/(v0^2*Cos[\[Theta]]^2))*x^2 //. {g -> 
     9.8, v0 -> 45, \[Theta] -> \[Pi]/3};
(*无阻力轨迹方程*)
Plot[{yyx[x], yy0[x]}, {x, 0, 200}, 
 PlotStyle -> Thickness[0.01], PlotRange -> {{0, 200}, {0, 100}}, 
 Frame -> True, FrameLabel -> {Style["x/m", 18], Style["y/m", 18]}, 
 FrameStyle -> Thickness[0.003], 
 PlotLegends -> {Style["有阻力", 18], Style["无阻力", 18]}]
(*绘图*)
\end{lstlisting}

\subsection{运行结果}

运行得到的代数函数如下:

\begin{equation}
	y(x)=-\frac{g m^2 \log \left(\frac{m \text{v0} \cos (\theta )}{m \text{v0} \cos (\theta )-b x}\right)}{b^2}+\frac{g m^2}{b^2}-\frac{g m \sec (\theta ) (m \text{v0} \cos (\theta )-b x)}{b^2 \text{v0}}+\frac{m \text{v0} \sin (\theta )}{b}-\frac{\tan (\theta ) (m \text{v0} \cos (\theta )-b x)}{b}.
\end{equation}

代入预设参数,得到结果(并与无阻力情况对比)如图1所示:

\begin{figure}[h]
 \center{\includegraphics[width=10cm]{hw81.png}}
 \caption{Problem 1 运行结果}
\end{figure}

\section{Problem 2}
\subsection{程序设定}

\textbf{问题描述:}

考察一维量子力学问题,质量为$m$的一个粒子束缚在势阱$V(x)$中运动,$V(x)$的形式如下:

\begin{equation}
	\begin{aligned}
		V(x)
			=\left\{
		\begin{array}{lr}
			-V_0,& |x|<a\\
			0,& |x|\le a
		\end{array},
	\right.
	\end{aligned}
\end{equation}
给出定态情形下本征波函数$\psi(x)$与本征能量$E$,画出前三个能级本征波函数。

\textbf{程序设定:}

\quad 使用自然单位制,

\quad $m=1GeV$,
\quad $V_0=1GeV$,
\quad $a=3GeV^{-1}$.

\subsection{程序实现}

程序实现如下:

\begin{lstlisting}[language=Mathematica]
Clear["Global`*"]

V0 = 1;
m = 1;
a = 3;
(*参数赋值*)

q[E0_] = Sqrt[2 m (V0 + E0)];
k[E0_] = Sqrt[-2 m*E0];
B0[E0_] = Exp[k[E0]*a]*Cos[q[E0]*a];
D0[E0_] = -Exp[k[E0]*a]*Sin[q[E0]*a];
(*已知条件带入*)

Uevenc[E0_, x_] = Cos[q[E0]*x];
Uevenp[E0_, x_] = B0[E0]*Exp[-k[E0]*x];
Uevenn[E0_, x_] = B0[E0]*Exp[k[E0]*x];
Uoddc[E0_, x_] = Sin[q[E0]*x];
Uoddp[E0_, x_] = -D0[E0]*Exp[-k[E0]*x];
Uoddn[E0_, x_] = D0[E0]*Exp[k[E0]*x];
(*写出通解*)

soleven = 
  Flatten[FindRoot[(D[Uevenc[E0, x], x] /. 
          x -> a) == (D[Uevenp[E0, x], x] /. x -> a) // 
       Simplify, {E0, #}] & /@ (Range[1, 10, 1]*(-10^-1))];
solodd = Flatten[
   FindRoot[(D[Uoddc[E0, x], x] /. x -> a) == (D[Uoddp[E0, x], x] /. 
          x -> a) // Simplify, {E0, #}] & /@ (Range[1, 10, 
       1]*(-10^-1))];
(*带入边界条件,求解超越方程*)

E3 = E0 /. soleven[[1]];
E2 = E0 /. solodd[[3]];
E1 = E0 /. soleven[[7]];
(*本征值能量获取*)

U1[x_] = Piecewise[{{Uevenn[E1, x], 
     x <= -a}, {Uevenc[E1, x], -a < x <= a}, {Uevenp[E1, x], 
     x > a}}];
U2[x_] = Piecewise[{{Uoddn[E2, x], 
     x <= -a}, {Uoddc[E2, x], -a < x <= a}, {Uoddp[E2, x], x > a}}];
U3[x_] = Piecewise[{{Uevenn[E3, x], 
     x <= -a}, {Uevenc[E3, x], -a < x <= a}, {Uevenp[E3, x], 
     x > a}}];
(*带入参数,获得未归一化波函数*)

A1 = Integrate[U1[x]*U1[x], {x, -Infinity, +Infinity}];
A2 = Integrate[U2[x]*U2[x], {x, -Infinity, +Infinity}];
A3 = Integrate[U3[x]*U3[x], {x, -Infinity, +Infinity}];
(*计算归一化因子*)

U10[x_] = PiecewiseExpand[U1[x]/(Sqrt[A1])]
(*归一化*)
Plot[U10[x], {x, -10, 10}, PlotStyle -> Thickness[0.01], 
 Frame -> True, FrameLabel -> {Style["x", 18], Style["\[Psi]", 18]}, 
 FrameStyle -> Thickness[0.003], 
 PlotLegends -> Style[ToString[E1] "=E1", 18]]
(*绘图*)
U20[x_] = PiecewiseExpand[U2[x]/(Sqrt[A2])]
(*归一化*)
Plot[U20[x], {x, -10, 10}, PlotStyle -> Thickness[0.01], 
 Frame -> True, FrameLabel -> {Style["x", 18], Style["\[Psi]", 18]}, 
 FrameStyle -> Thickness[0.003], 
 PlotLegends -> Style[ToString[E2] "=E2", 18]]
(*绘图*)
U30[x_] = PiecewiseExpand[U3[x]/(Sqrt[A3])]
(*归一化*)
Plot[U30[x], {x, -10, 10}, PlotStyle -> Thickness[0.01], 
 Frame -> True, FrameLabel -> {Style["x", 18], Style["\[Psi]", 18]}, 
 FrameStyle -> Thickness[0.003], 
 PlotLegends -> Style[ToString[E3] "=E3", 18]]
(*绘图*)
\end{lstlisting}

\subsection{运行结果}

给出的3个本征态分别如下:

\begin{equation}
\begin{aligned}
		E_1=-0.910757GeV,\quad
		\psi(x)
			=\left\{
		\begin{array}{lr}
		8.85551 e^{1.34963 x}, & x\le -3\\
		 0.517023 \cos (0.422476 x),& -3<x\le 3\\
			 8.85551 e^{-1.34963 x},& x>3
		\end{array}.
	\right.
	\end{aligned}
\end{equation}

\begin{equation}
\begin{aligned}
		E_2=-0.650311GeV,\quad
		\psi(x)
			=\left\{
		\begin{array}{lr}
		-9.19332 e^{1.14045 x}, & x\le -3\\
		 0.507879 \sin (0.836289 x),& -3<x\le 3\\
			9.19332 e^{-1.14045 x},& x>3
		\end{array}.
	\right.
	\end{aligned}
\end{equation}

\begin{equation}
\begin{aligned}
		E_3=-0.252509GeV,\quad
		\psi(x)
			=\left\{
		\begin{array}{lr}
		-3.47226 e^{0.710646 x}, & x\le -3\\
		0.476343 \cos (1.22269 x),& -3<x\le 3\\
			-3.47226 e^{-0.710646 x},& x>3
		\end{array}.
	\right.
	\end{aligned}
\end{equation}

绘出三个本征态波函数如图2所示。

\begin{figure}[tbp]
 \subfigure[本征态1]{\includegraphics[width=5cm]{hw821.eps}}
 \subfigure[本征态2]{\includegraphics[width=5cm]{hw822.eps}}
 \subfigure[本征态3]{\includegraphics[width=5cm]{hw823.eps}}
 \caption{Problem 2 运行结果}
\end{figure}


\end{document}	
