%!TEX program = xelatex
\documentclass{article}
\usepackage[UTF8]{ctex}
\usepackage{authblk}
\usepackage{listings}
\usepackage{subfigure}
\usepackage{amsmath, amsfonts, amsthm, amssymb, graphicx, xcolor, hyperref, mathrsfs}
\lstset{
    basicstyle          =   \sffamily,          % 基本代码风格
    keywordstyle        =   \bfseries,          % 关键字风格
    commentstyle        =   \rmfamily\itshape,  % 注释的风格,斜体
    stringstyle         =   \ttfamily,  % 字符串风格
    flexiblecolumns,                % 别问为什么,加上这个
    numbers             =   left,   % 行号的位置在左边
    showspaces          =   false,  % 是否显示空格,显示了有点乱,所以不现实了
    numberstyle         =   \zihao{-5}\ttfamily,    % 行号的样式,小五号,tt等宽字体
    showstringspaces    =   false,
    captionpos          =   t,      % 这段代码的名字所呈现的位置,t指的是top上面
    frame               =   lrtb,   % 显示边框
    breaklines%自动换行
}

\title{计算物理作业7} %文章标题
\author[a]{王一杰} %作者的名称
\affil[a]{中国科学技术大学}
\date{\today}%日期
\usepackage [a4paper, left=25mm,right=25mm,top=20mm,bottom=20mm]{geometry} %设置页面的环境,a4纸张大小,左右上下边距信息

\begin{document}
\maketitle
%添加这一句才能够显示标题等信息
%\tableofcontents

\section{Problem 1}
\subsection{程序设定}
\textbf{问题描述:}编写用Verlet速度算法求解三维分子运动方程的程序。

\textbf{程序设定:}

\quad 计算步长$h=0.001$,

\quad 计算步数$n=64000$,

\quad 势能构造$V=-\frac{1}{r}-\frac{1}{|\overrightarrow{r}-\overrightarrow{r_0}|},\ where\  \overrightarrow{r_0}=(1,1,1)$,即具有两个力心的平方反比吸引力。

\quad 分子初始参数:

\qquad 质量:$m=1$

\qquad 初速度:$(v_x,v_y,v_z)=(-1,0.1,0.5)$,

\qquad 初始位置:$(x,y,z)=(1,0,0.5)$,

\subsection{程序实现}

程序实现如下:

\begin{lstlisting}[language=MATLAB]
n=64000;					%计算步数
h=0.001;					%计算步长
x=zeros(n+1,3);					%位置储存单元
v=zeros(n+1,3);					%速度储存单元
x(1,1)=1;					%位置初始化
x(1,2)=0;
x(1,3)=0.5;
v(1,1)=-1;					%速度初始化
v(1,2)=0.1;
v(1,3)=0.5;
F=zeros(2,3);					%作用力存储单元
for i=1:1:n					%演化计算
   for j=1:1:3
       F(1,j)=-x(i,j)/(x(i,1)^2+x(i,2)^2+x(i,3)^2)^1.5-(x(i,j)-1)/((x(i,1)-1)^2+(x(i,2)-1)^2+(x(i,3)-1)^2)^1.5;				%F_i计算
       x(i+1,j)=x(i,j)+h*v(i,j)+0.5*F(1,j)*h^2;		%x_i+1计算
   end
   for j=1:1:3
       F(2,j)=-x(i+1,j)/(x(i+1,1)^2+x(i+1,2)^2+x(i+1,3)^2)^1.5-(x(i+1,j)-1)/((x(i+1,1)-1)^2+(x(i+1,2)-1)^2+(x(i+1,3)-1)^2)^1.5;		%F_i+1计算
       v(i+1,j)=v(i,j)+0.5*h*(F(2,j)+F(1,j));		%V_i+1计算
   end
end
\end{lstlisting}

\subsection{运行结果}

演化结果如图1所示,计算结果期望相符,图片分别展示了两个力心的轨迹和单个力心的轨迹(即$V=-1/r$),与预期相符。可见,计算结果正确。

\begin{figure}[tbp]
 \subfigure[双力心轨迹]{\includegraphics[width=8cm]{hw712.eps}}
 \subfigure[单力心轨迹]{\includegraphics[width=8cm]{hw713.eps}}
 \caption{Problem 1 运行结果}
\end{figure}

\section{Problem 2}
\subsection{程序设定}

总能量固定的单原子系统的分子动力学模拟。

元胞为$L_x=L_y=L_z=10$,划分为$10\times 10\times 10$的正方形网格,

元胞内原子数$N=64$,

原子质量$m=1$,

位势为Lenard-Jones势,其中$\varepsilon=\sigma=1$,

边界条件为周期性边界条件,初始位置是随机分布在正则节点上,

初始速度为按$[-1,1]$随机分布。

分子动力学模拟步长取为$\Delta t=0.02$,

模拟$100\sim200$步后原子的速度分布和位置分布。

\subsection{程序实现}

\begin{lstlisting}[language=MATLAB]
N=64;				%粒子数
step=200;			%计算步数
h=0.02;				%计算步长
x=zeros(N,3);			%位置储存单元
x1=zeros(N,3);			%位置临时储存单元
xr=zeros(N,3);			%相对位置储存单元
flag=0;
E=zeros(step,1);		%Energy
T=2;				%温度设定
rc=4;				%截断半径

l=1;
while(l<N+1)			%位置随机生成,此处需注意一个格点只能生成一个粒子,否则会出现nan
    flag=0;
    x(l,:)=round(0.5+9*rand(1,3));
    for k=1:1:l-1
       if (x(k,1)==x(l,1))&&(x(k,2)==x(l,2))&&(x(k,3)==x(l,3))
          flag=1;
       end
    end
    if flag==1
        l=l-1;
    end
    l=l+1;
end
v=(2*rand(N,3)-1);		%速度随机生成
F1=zeros(64,3);			%作用力存储单元
F2=zeros(64,3);			%作用力存储单元

for i=1:1:step			%演化过程
   for n=1:1:N
       for r=1:1:3		%作用力的初始置0
           F1(n,r)=0;
           F2(n,r)=0;
       end
       for k=1:1:N		%最小像力约定下的相对位置计算
           if k~=n
               for m=1:1:3
                    if x(n,m)-x(k,m)>5
                        xr(k,m)=x(k,m)+10;
                        xr(k,m)=xr(k,m)-x(n,m);
                    elseif x(n,m)-x(k,m)<=-5
                        xr(k,m)=x(k,m)-10;
                        xr(k,m)=xr(k,m)-x(n,m);
                    else
                        xr(k,m)=x(k,m)-x(n,m);
                    end
               end
               for m=1:1:3	%最小像力约定下的相互作用力计算
                   F1(n,m)=F1(n,m)-48*xr(k,m)/(xr(k,1)^2+xr(k,2)^2+xr(k,3)^2)^7+24*xr(k,m)/(xr(k,1)^2+xr(k,2)^2+xr(k,3)^2)^4;
                if (xr(k,1)^2+xr(k,2)^2+xr(k,3)^2)>rc^2
                	F1(n,m)=F1(n,m)+48*xr(k,m)/(xr(k,1)^2+xr(k,2)^2+xr(k,3)^2)^7-24*xr(k,m)/(xr(k,1)^2+xr(k,2)^2+xr(k,3)^2)^4;	%截断修正
                end
               end
           end
       end
       for m=1:1:3		%位置演化
          x1(n,m)=mod(x(n,m)+h*v(n,m)+0.5*F1(n,m)*h^2,10); 
       end
   end
   
   for n=1:1:N
       for k=1:1:N		%最小像力约定下的新时刻相对位置计算
           if k~=n
               for m=1:1:3
                    if x1(n,m)-x1(k,m)>5
                        xr(k,m)=x1(k,m)+10;
                        xr(k,m)=xr(k,m)-x1(n,m);
                    elseif x1(n,m)-x1(k,m)<=-5
                        xr(k,m)=x1(k,m)-10;
                        xr(k,m)=xr(k,m)-x1(n,m);
                    else
                        xr(k,m)=x1(k,m)-x1(n,m);
                    end
               end
           
               for m=1:1:3	%最小像力约定下的新时刻相互作用力计算
                    F2(n,m)=F2(n,m)-48*xr(k,m)/(xr(k,1)^2+xr(k,2)^2+xr(k,3)^2)^7+24*xr(k,m)/(xr(k,1)^2+xr(k,2)^2+xr(k,3)^2)^4;
                    if (xr(k,1)^2+xr(k,2)^2+xr(k,3)^2)>rc^2
                        F2(n,m)=F2(n,m)+48*xr(k,m)/(xr(k,1)^2+xr(k,2)^2+xr(k,3)^2)^7-24*xr(k,m)/(xr(k,1)^2+xr(k,2)^2+xr(k,3)^2)^4;	%截断修正
                    end
               end
           end
       end
       for m=1:1:3		%速度演化
          v(n,m)=v(n,m)+0.5*(F1(n,m)+F2(n,m))*h; 
       end
   end
   beta=0;
   for n=1:1:N
       for m=1:1:3
          beta=beta+v(n,m)^2; 
       end
   end
   beta=(T*(N-1)/(beta*16))^0.5;		%速度约化因子计算
   for n=1:1:N
       for m=1:1:3
          v(n,m)=beta*v(n,m); 			%速度修正
       end
   end
   for n=1:1:N
      for m=1:1:3
         x(n,m)=x1(n,m); 			%位置临时储存单元输入进位置储存单元
      end
   end
   for n=1:1:N
       for m=1:1:3
          E(i)=E(i)+v(n,m)^2; 			%动能计算
       end
   end
   E(i)=E(i)/2;
   %---------------------------以下用于监视演化过程
     %  figure(1)
     %  plot3(v(:,1),v(:,2),v(:,3),'o','MarkerSize',5,'MarkerFaceColor','r')
     %  xlim([-0.4,0.4])
     %  ylim([-0.4,0.4])
     %  zlim([-0.4,0.4])
     %  grid on
     %  figure(2)
     %  plot3(x(:,1),x(:,2),x(:,3),'o','MarkerSize',5,'MarkerFaceColor','r')
     %  xlim([0,10])
     %  ylim([0,10])
     %  zlim([0,10])
     %  grid on
     %  pause(0.01)
   %---------------------------演化监视绘图结束
end
figure(1)					%速度相空间绘制
plot3(v(:,1),v(:,2),v(:,3),'o','MarkerSize',5,'MarkerFaceColor','r')
xlim([-1,1])
ylim([-1,1])
zlim([-1,1])
xlabel('V_x')
ylabel('V_y')
zlabel('V_z')
set(gca,'FontSize',14)
grid on

figure(2)					%位置相空间绘制
plot3(x(:,1),x(:,2),x(:,3),'o','MarkerSize',5,'MarkerFaceColor','r')
xlim([0,10])
ylim([0,10])
zlim([0,10])
xlabel('x')
ylabel('y')
zlabel('z')
set(gca,'FontSize',14)
grid on
\end{lstlisting}

演化结果如图2所示,分别展现了速度相空间和位置相空间的原子分布,计算结果期望相符(速度满足Maxwell速度分布率,位置均匀分布)。可见,计算结果正确。

\begin{figure}[tbp]
 \subfigure[速度相空间]{\includegraphics[width=8cm]{vspace.eps}}
 \subfigure[位置相空间]{\includegraphics[width=8cm]{xspace.eps}}
  \caption{Problem 2 运行结果}
\end{figure}
\end{document}	
