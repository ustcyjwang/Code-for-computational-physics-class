%!TEX program = xelatex
\documentclass{article}
\usepackage[UTF8]{ctex}
\usepackage{authblk}
\usepackage{listings}
\usepackage{subfigure}
\usepackage{amsmath, amsfonts, amsthm, amssymb, graphicx, xcolor, hyperref, mathrsfs}
\lstset{
    basicstyle          =   \sffamily,          % 基本代码风格
    keywordstyle        =   \bfseries,          % 关键字风格
    commentstyle        =   \rmfamily\itshape,  % 注释的风格,斜体
    stringstyle         =   \ttfamily,  % 字符串风格
    flexiblecolumns,                % 别问为什么,加上这个
    numbers             =   left,   % 行号的位置在左边
    showspaces          =   false,  % 是否显示空格,显示了有点乱,所以不现实了
    numberstyle         =   \zihao{-5}\ttfamily,    % 行号的样式,小五号,tt等宽字体
    showstringspaces    =   false,
    captionpos          =   t,      % 这段代码的名字所呈现的位置,t指的是top上面
    frame               =   lrtb,   % 显示边框
}

\title{计算物理作业6} %文章标题
\author[a]{王一杰} %作者的名称
\affil[a]{中国科学技术大学}
\date{\today}%日期
\usepackage [a4paper, left=20mm,right=20mm,top=10mm,bottom=15mm]{geometry} %设置页面的环境,a4纸张大小,左右上下边距信息

\begin{document}
\maketitle
%添加这一句才能够显示标题等信息
%\tableofcontents

\section{Homework 6}
\subsection{Problem 1}

采用有限差分法(取$h=\frac{\pi}{90}$)并结合代数线性方程组求解的超松弛迭代法(取$\omega=\frac{7}{4}$,迭代1000次),数值求解拉普拉斯边值问题的程序如下,很容易写出:
\begin{lstlisting}[language=MATLAB]
N=90;							%取格点间距为pi/90
omega=7/4;						%超松弛迭代参数
x=0:pi/N:pi;						%x格点
y=0:pi/N:pi;						%y格点
phi=zeros(N+1);						%phi存储矩阵
phi(:,N+1)=sin(x);					%边界条件
temp=0;
for n=1:1:1000						%迭代1000次
    for i=2:1:N						%内部格点计算
        for j=2:1:N
            temp=(phi(i+1,j)+phi(i,j+1)+phi(i-1,j)+phi(i,j-1))/4;
            phi(i,j)=(1-omega)*phi(i,j)+omega*temp;
        end
    end
end
\end{lstlisting}

演化结果如图1所示,计算结果期望相符,三幅图分别展示了$\varphi$的计算值,$\varphi$的理论值,和理论与计算值的绝对偏差$\delta$(已达到$10^{-5}$量级)。可见,计算结果是正确的,且超松弛迭代计算收敛非常快。





\begin{figure}[tbp]
 \subfigure[$\phi$的计算值]{\includegraphics[width=5.5cm]{phi.eps}}
 \subfigure[$\phi$的理论值]{\includegraphics[width=5.5cm]{the.eps}}
 \subfigure[$\phi$的误差$\delta$分布]{\includegraphics[width=5.5cm]{delta.eps}}
 \caption{Problem1的程序计算结果}
\end{figure}

\end{document}	