%!TEX program = xelatex
\documentclass{article}
\usepackage[UTF8]{ctex}
\usepackage{authblk}
\usepackage{listings}
\usepackage{subfigure}
\usepackage{amsmath, amsfonts, amsthm, amssymb, graphicx, xcolor, hyperref, mathrsfs}
\lstset{
    basicstyle          =   \sffamily,          % 基本代码风格
    keywordstyle        =   \bfseries,          % 关键字风格
    commentstyle        =   \rmfamily\itshape,  % 注释的风格,斜体
    stringstyle         =   \ttfamily,  % 字符串风格
    flexiblecolumns,                % 别问为什么,加上这个
    numbers             =   left,   % 行号的位置在左边
    showspaces          =   false,  % 是否显示空格,显示了有点乱,所以不现实了
    numberstyle         =   \zihao{-5}\ttfamily,    % 行号的样式,小五号,tt等宽字体
    showstringspaces    =   false,
    captionpos          =   t,      % 这段代码的名字所呈现的位置,t指的是top上面
    frame               =   lrtb,   % 显示边框
}

\title{计算物理作业3} %文章标题
\author[a]{王一杰} %作者的名称
\affil[a]{中国科学技术大学}
\date{\today}%日期
\usepackage [a4paper, left=25mm,right=25mm,top=15mm,bottom=15mm]{geometry} %设置页面的环境,a4纸张大小,左右上下边距信息

\begin{document}
\maketitle
%添加这一句才能够显示标题等信息
%\tableofcontents

\section{Homework 3}
\subsection{Problem 1}
该问题的程序很容易给出如下(MATLAB代码):
\begin{lstlisting}[language=MATLAB]
N=10000;					%粒子数量
steps=3.*10.^4;					%演化步数
state=zeros(1,N);				%粒子状态记录,1代表R,0代表L
statesmemory=zeros(1,steps+1);    		%记录每次右侧粒子数量
statesmemory(1,1)=sum(state);
Nstep=1:1:steps+1;
for i=1:steps
   evolutionparticle=round(rand(1,1)*N+0.5);	%随机演化
    state(1,evolutionparticle)=rem(1+state(1,evolutionparticle),2);
    statesmemory(1,1+i)=sum(state);		%记录每次右侧粒子数量
end
Lstatesmemory=N-statesmemory;			%计算每次演化左侧粒子数量
\end{lstlisting}
演化结果如图1(a)所示,右侧粒子数量理论计算期望值是$n_R=\frac{N}{2}(1-(\frac{2}{N})^{n})$,其中$N$为粒子总数,$n$为演化次数,计算结果与理论计算期望相符。
\begin{figure}[tbp]
  \subfigure[Problem1结果]{\includegraphics[width=8cm]{pro1.eps}}
  \subfigure[Problem2结果]{\includegraphics[width=8cm]{pro2.eps}}
 \caption{程序计算结果}
\end{figure}

\subsection{Problem 2}
由于最大速率$V_{max}=100m/s\ll c$,故暂不计算相对论效应,且假设分子质量相同。

第一步是随机选择两个发生相互作用的分子,这可以用均匀随机变量确定。

第二步是计算质心速率,若入射速率记为$v_i,v_j$,易于得到质心的速率为:$v_c=\frac{\sqrt{v_i^2+v_j^2+2v_iv_jcos\theta_i}}{2}$,其中$\theta_i$为两个分子入射夹角,在三维情况下其分布函数同$f(\theta_i)=\frac{1}{2}sin(\theta_i)$,可以用$[0,1]$区间的均匀随机变量$x$由变换$\theta_i=arccos(2x-1)$确定。

第三步,由于分子质量相同,相对质心速率相等且为:$v_{cr}=\frac{\sqrt{v_i^2+v_j^2-2v_iv_jcos\theta_i}}{2}$。

第四步,易于得到出射速率为$v_{o_1}=\sqrt{v_{cr}^2+v_c^2+2v_{cr}v_ccos\theta_o}$和$v_{o_2}=\sqrt{v_{cr}^2+v_c^2-2v_{cr}v_ccos\theta_o}$,其中$\theta_o$为两个分子质心出射夹角,在三维情况下其分布函数同$f(\theta_o)=\frac{1}{2}sin(\theta_o)$,可以用$[0,1]$区间的均匀随机变量$x$由变换$\theta_o=arccos(2x-1)$确定。

上述流程的程序很容易如下(MATLAB代码):
\begin{lstlisting}[language=MATLAB]
N=1000;							%分子数量
num=1:1:N;						%分子序号
v=num*0.1;						%分子初始速率
steps=10.^6;						%演化步数
for i=1:1:steps
    thetai=acos(rand(1,1)*2-1);				%入射角
    evolutionparticle1=round(rand(1,1)*N+0.5); 		%分子选择1
    evolutionparticle2=round(rand(1,1)*N+0.5); 		%分子选择2
    v1=v(1,evolutionparticle1);				%分子1速率读取
    v2=v(1,evolutionparticle2);				%分子2速率读取
    vc=sqrt(v1^2+v2^2+2*v1*v2*cos(thetai))/2;		%分子质心速率
    thetacout=acos(rand(1,1)*2-1);
    vcr=sqrt(v1^2+v2^2-2*v1*v2*cos(thetai))/2;		%分子相对质心速率
    vo1=sqrt(vcr^2+vc^2+2*vcr*vc*cos(thetacout));	%分子1出射速率
    vo2=sqrt(vcr^2+vc^2+2*vcr*vc*cos(pi-thetacout));	%分子2出射速率
    v(1,evolutionparticle1)=vo1;			%分子1速率存储
    v(1,evolutionparticle2)=vo2;			%分子2速率存储
end
hist(v,20);

\end{lstlisting}

Maxwell速率分布率给出的分子速率分布结果是$N(v)=4\pi v^{2}\left(\frac{1}{2\pi\alpha}\right)^{\frac{3}{2}}e^{-\frac{v^{2}}{2\alpha}}\cdot N$,其中$v$为分子速率,$N$为分子总数,$\alpha$满足$\alpha=\frac{1}{3N}\sum_{i=1}^N v_i^2$。

演化结果如图2(b)所示,可见演化结果与Maxwell速率分布率基本吻合,可见程序运行结果良好。
\end{document}	00000