%!TEX program = xelatex
\documentclass{article}
\usepackage[UTF8]{ctex}
\usepackage{authblk}
\usepackage{listings}
\usepackage{subfigure}
\usepackage{amsmath, amsfonts, amsthm, amssymb, graphicx, xcolor, hyperref, mathrsfs}
\lstset{
    basicstyle          =   \sffamily,          % 基本代码风格
    keywordstyle        =   \bfseries,          % 关键字风格
    commentstyle        =   \rmfamily\itshape,  % 注释的风格,斜体
    stringstyle         =   \ttfamily,  % 字符串风格
    flexiblecolumns,                % 别问为什么,加上这个
    numbers             =   left,   % 行号的位置在左边
    showspaces          =   false,  % 是否显示空格,显示了有点乱,所以不现实了
    numberstyle         =   \zihao{-5}\ttfamily,    % 行号的样式,小五号,tt等宽字体
    showstringspaces    =   false,
    captionpos          =   t,      % 这段代码的名字所呈现的位置,t指的是top上面
    frame               =   lrtb,   % 显示边框
}

\title{计算物理作业1-2} %文章标题
\author[a]{王一杰} %作者的名称
\affil[a]{中国科学技术大学}
\date{2021.10.02}%日期
\usepackage [a4paper, left=25mm,right=25mm,top=25mm,bottom=25mm]{geometry} %设置页面的环境,a4纸张大小,左右上下边距信息

\begin{document}
\maketitle
%添加这一句才能够显示标题等信息
%\tableofcontents

\section{Homework 1}
若$(x,x+dx)$映射到区间$(y,y+dy)$,且区间概率不变,则有$f(x)dx=g(y)dy$。容易导出$f(x)=g(y)|\frac{dy}{dx}|$或$g(y)=f(x)|\frac{dx}{dy}|$。
\section{Homework 2}
\subsection{Problem 1}
(1)目标函数为$f ( x ) = \lambda e ^ { - \lambda x } , \quad x \in \lbrack 0 , + \infty )  $,若变换前的随机变量$y$为$[0,1]$均匀分布,有$g(y)=1$,则$|\frac{dy}{dx}|=\lambda e^{-\lambda x}$,可以解得:$x=-\frac{1}{\lambda}ln(y)$。

由此可以给出实现代码如下(MATLAB代码):
\begin{lstlisting}[language=MATLAB]
N=10000;
lambda=1;
s1=rand(1,N);
x=-(1/lambda)*log(s1);
hist(x,100);
\end{lstlisting}
效果如图1(a)所示。

\begin{figure}[tbp]
  \subfigure[Problem1结果]{\includegraphics[width=8cm]{p2-1.eps}}
  \subfigure[Problem2结果]{\includegraphics[width=8cm]{p2-2.eps}}
 \caption{事例产生器生成随机序列频次统计图}
\end{figure}

(2)由于$I=\int_0^{+\infty}x^{\frac{3}{2}}e^{-x}dx=\int_0^{+\infty}h(x)f(x)dx$,其中$f(x)=e^{-x}$可以由(1)中方法生成满足该分布密度的随机变量,$h(x)=x^{\frac{3}{2}}$。则有$I$的估计值$I=\frac{1}{n}\Sigma_{i=1}^n h(x_i)$
由此可以给出实现代码如下(MATLAB代码):
\begin{lstlisting}[language=MATLAB]
N=10000000; %此处取N=10000000
lambda=1;
s1=rand(1,N);
x=-(1/lambda)*log(s1);
answer=sum(x.^(1.5))/N
\end{lstlisting}
给出计算结果为$I=1.329\pm0.002\ (P=0.997)$,其中计算值由上述代码给出,不确定度由理论计算公式$\sigma=\sqrt{\frac{V(h)}{n}},where\ V ( h ) = \int _ { 0 } ^ { +\infty } ( h(x) - I ) ^ { 2 } f ( x ) d x  $计算。
\subsection{Problem 2}
(1)目标函数为$f ( x ) = \frac { \Gamma } { \pi } \frac { 1 } { ( x - x _ { 0 } ) ^ { 2 } + \Gamma ^ { 2 } }   $,若变换前的随机变量$y$为$[0,1]$均匀分布,有$g(y)=1$,则$|\frac{dy}{dx}|=\frac { \Gamma } { \pi } \frac { 1 } { ( x - x _ { 0 } ) ^ { 2 } + \Gamma ^ { 2 } }$,可以解得:$x=\Gamma\cdot tan(\pi(y-\frac{1}{2}))+x_0$。

由此可以给出实现代码如下(MATLAB代码):
\begin{lstlisting}[language=MATLAB]
N=10000;
x0=0;
gamma=1;
s1=rand(1,N);
x=gamma*tan(pi*(s1-0.5))+x0;
xbins=-22:22;
hist(x,xbins,44);
axis([-20 20 0 inf])%此处仅展示[-20,20]区间统计频次
\end{lstlisting}
效果如图1(b)所示。

(2)由于$I=I = \int _ { 0 } ^ { + \infty } \frac { \sqrt { x } } { 1 + x ^ { 2 } } d x  =\int_0^{+\infty}h(x)f(x)dx$,其中$f(x)=\frac { 1 } { \pi } \frac { 1 } { x^ { 2 } +1 }$可以由(1)中方法生成满足该分布密度的随机变量,$h(x)=\pi\cdot\sqrt{x}$。则有$I$的估计值$I=\frac{1}{n}\Sigma_{i=1}^n h(x_i)$
由此可以给出实现代码如下(MATLAB代码):
\begin{lstlisting}[language=MATLAB]
N=10000000; %此处取N=10000000
x0=0;
gamma=1;
s1=rand(1,N);
x=gamma*tan(pi*(s1-0.5))+x0;
answer=real(sum(pi*x.^(0.5))/N) %此处利用取实部舍去x<0部分
\end{lstlisting}
给出计算结果为$I=2.220\pm0.009\ (P=0.997)$,其中计算值由上述代码给出,不确定度由理论计算公式$\sigma=\sqrt{\frac{V(h)}{n_{eff}}},where\ V ( h ) = \int _ { 0 } ^ { +\infty } ( h(x) - I ) ^ { 2 } f ( x ) d x  $计算,而且$n_eff\approx\frac{1}{2}n$,因为生产随机数只有$x\geq0$的部分是有效的。
\subsection{Problem3}
由于若$\xi$为$[0,1]$上均匀分布,且有$\xi=e^{-ax}$,则$\eta=\Sigma_{i=1}^{n}x_i$,且$\{x_i\}$为服从$\rho(x)=a\cdot e^{-ax}$的指数分布,也是$\Gamma(1,a)$分布。由于$\Gamma$分布有可加性原则(可以将两个服从$Gamma$分布的变量$x,\ y$化为$u=x+y$,$v=y$,然后利用联合密度分布函数计算$u$的边缘分布即可)。即若$x_i$服从$x_i$服从$\Gamma(\alpha,a)$,$x_j$服从$\Gamma(\beta,a)$分布,则$x_i+x_j$服从$x_i$服从$\Gamma(\alpha+\beta,a)$,所以由于该题中,任意$x_i$服从$\Gamma(1,a)$分布,故$\eta=\Sigma_{i=1}^{n}x_i$服从$\Gamma(n,a)$分布,有$f ( x )  = \frac { a ^ { n } } { ( n - 1 ) ! } x ^ { n - 1 } e ^ { - a x }   $,命题得证。
\subsection{Problem4}
易于导出:$P(\xi<x)=(2\pi)^{-n/2}\int\cdots\int_B e^{-\frac{1}{2}\Sigma_{i=1}^{n}x_i^2}dx_1\cdots dx_n$,其中B是n维球内空间,在秋坐标变换下有$P(\xi<x)=\frac{2\pi^{n/2}}{\Gamma(n/2)}\cdot\frac{1}{(2\pi)^{n/2}}\int_0^{\sqrt{x}}e^{-\frac{1}{2}r^2}r^{n-1}dr$,则$f(x)=\frac{d}{dx}P(\xi<x)= \frac{1}{2^{n/2}\Gamma(n/2)}x^{\frac{n}{2}-1}e^{-x/2}$,命题得证。
\end{document}	
